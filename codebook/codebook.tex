\documentclass[10pt,twocolumn,oneside]{article}
\setlength{\columnsep}{10pt}                                                                    %兩欄模式的間距
\setlength{\columnseprule}{0pt}                                                                %兩欄模式間格線粗細

\usepackage{amsthm}								%定義,例題
\usepackage{amssymb}
%\usepackage[margin=2cm]{geometry}
\usepackage{fontspec}								%設定字體
\usepackage{color}
\usepackage[x11names]{xcolor}
\usepackage{xeCJK}								%xeCJK
\usepackage{listings}								%顯示code用的
%\usepackage[Glenn]{fncychap}						%排版,頁面模板
\usepackage{fancyhdr}								%設定頁首頁尾
\usepackage{graphicx}								%Graphic
\usepackage{enumerate}
\usepackage{titlesec}
\usepackage{amsmath}
\usepackage[CheckSingle, CJKmath]{xeCJK}
% \usepackage{CJKulem}

%\usepackage[T1]{fontenc}
\usepackage{amsmath, courier, listings, fancyhdr, graphicx}
\topmargin=0pt
\headsep=5pt
\textheight=780pt
\footskip=0pt
\voffset=-40pt
\textwidth=545pt
\marginparsep=0pt
\marginparwidth=0pt
\marginparpush=0pt
\oddsidemargin=0pt
\evensidemargin=0pt
\hoffset=-42pt

%\renewcommand\listfigurename{圖目錄}
%\renewcommand\listtablename{表目錄} 

%%%%%%%%%%%%%%%%%%%%%%%%%%%%%

\setmainfont{Consolas}				%主要字型
\setmonofont{Monaco}				%主要字型
\setCJKmainfont{Source Han Sans TC}
% \setCJKmainfont{Consolas}			%中文字型
%\setmainfont{sourcecodepro}
\XeTeXlinebreaklocale "zh"						%中文自動換行
\XeTeXlinebreakskip = 0pt plus 1pt				%設定段落之間的距離
\setcounter{secnumdepth}{3}						%目錄顯示第三層

%%%%%%%%%%%%%%%%%%%%%%%%%%%%%
\makeatletter
\lst@CCPutMacro\lst@ProcessOther {"2D}{\lst@ttfamily{-{}}{-{}}}
\@empty\z@\@empty
\makeatother
\lstset{											% Code顯示
language=C++,										% the language of the code
basicstyle=\footnotesize\ttfamily, 						% the size of the fonts that are used for the code
%numbers=left,										% where to put the line-numbers
numberstyle=\footnotesize,						% the size of the fonts that are used for the line-numbers
stepnumber=1,										% the step between two line-numbers. If it's 1, each line  will be numbered
numbersep=5pt,										% how far the line-numbers are from the code
backgroundcolor=\color{white},					% choose the background color. You must add \usepackage{color}
showspaces=false,									% show spaces adding particular underscores
showstringspaces=false,							% underline spaces within strings
showtabs=false,									% show tabs within strings adding particular underscores
frame=false,											% adds a frame around the code
tabsize=2,											% sets default tabsize to 2 spaces
captionpos=b,										% sets the caption-position to bottom
breaklines=true,									% sets automatic line breaking
breakatwhitespace=false,							% sets if automatic breaks should only happen at whitespace
escapeinside={\%*}{*)},							% if you want to add a comment within your code
morekeywords={*},									% if you want to add more keywords to the set
keywordstyle=\bfseries\color{Blue1},
commentstyle=\itshape\color{Red4},
stringstyle=\itshape\color{Green4},
}

%%%%%%%%%%%%%%%%%%%%%%%%%%%%%

\begin{document}
\pagestyle{fancy}
\fancyfoot{}
%\fancyfoot[R]{\includegraphics[width=20pt]{ironwood.jpg}}
\fancyhead[L]{NCTU Tmprry (2016/10/15)}
\fancyhead[R]{\thepage}
\renewcommand{\headrulewidth}{0.4pt}
\renewcommand{\contentsname}{Contents} 

\scriptsize
\tableofcontents
%%%%%%%%%%%%%%%%%%%%%%%%%%%%%

\newpage

\section{Basic}

\subsection{vimrc}
\lstinputlisting{Basic/vimrc}
\subsection{BigInt}
\lstinputlisting{Basic/Bigint.cpp}
\subsection{Random}
\lstinputlisting{Basic/Random.cpp}

%\newpage

\section{Mathmatics}

\subsection{Miller Rabin}
\lstinputlisting{Math/Miller-Rabin.cpp}
\subsection{ax+by=gcd(a,b)}
\lstinputlisting{Math/ax+by=gcd.cpp}
\subsection{FFT}
\lstinputlisting{Math/fft.cpp}
\subsection{Hash}
\lstinputlisting{Math/Hash.cpp}
\subsection{Convex Hull}
\lstinputlisting{Math/ConvexHull.cpp}
\subsection{Eratosthenes}
\lstinputlisting{Math/Eratosthenes.cpp}
\subsection{GaussElimination}
\lstinputlisting{Math/GaussElimination.cpp}
\subsection{Inverse}
\lstinputlisting{Math/Inverse.cpp}
\subsection{IterSet}
\lstinputlisting{Math/IterSet.cpp}
\subsection{LinearPrime}
\lstinputlisting{Math/LinearPrime.cpp}
\subsection{SG}
\lstinputlisting{Math/Sprague-Grundy.cpp}


\section{Geometry}


\section{Flow}

\subsection{Dinic}
\lstinputlisting{Flow/dinic.cpp}

\section{Graph}

\subsection{Strongly Connected Component(SCC)}
\lstinputlisting{Graph/Kosaraju_SCC.cpp}
\subsection{Euler Circuit}
\lstinputlisting{Graph/EulerCircuit.cpp}
\subsection{Hungarian}
\lstinputlisting{Graph/Hungarian.cpp}
\subsection{Maximum Clique}
\lstinputlisting{Graph/MaximumClique.cpp}
\subsection{Tarjan}
\lstinputlisting{Graph/Tarjan.cpp}

\subsection{LCA}
\lstinputlisting{Graph/LCA.cpp}

\section{Data Structure}

\subsection{Disjoint Set}
\lstinputlisting{DataStructure/djs.cpp}
\subsection{Sparse Table}
\lstinputlisting{DataStructure/SparseTable.h}
\subsection{Treap}
\lstinputlisting{DataStructure/Treap.cpp}

\section{String}

\subsection{KMP}
\lstinputlisting{String/KMP.h}
\subsection{AC}
\lstinputlisting{String/AC.cpp}
\subsection{Z-value}
\lstinputlisting{String/Z-value.cpp}

\section{Dark Code}

\subsection{輸入優化}
\lstinputlisting{DarkCode/IO_optimization.cpp}

\section{Search}


\section{Others}


\section{Persistence}

%\newpage

\end{document}
